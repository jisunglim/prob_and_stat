\documentclass[twoside]{article}
\setlength{\oddsidemargin}{0 in}
\setlength{\evensidemargin}{0 in}
\setlength{\topmargin}{-0.6 in}
\setlength{\textwidth}{6.5 in}
\setlength{\textheight}{8.5 in}
\setlength{\headsep}{0.75 in}
\setlength{\parindent}{0 in}
\setlength{\parskip}{0.1 in}

%
% ADD PACKAGES here:
%
\usepackage[utf8]{inputenc}
\usepackage{amsmath,amsfonts,amssymb,graphicx,mathtools}
\usepackage[document]{ragged2e}
\usepackage[Euler]{upgreek}
\usepackage{amsthm}

\usepackage[backend=biber]{biblatex}
\addbibresource{ref.bib}

\graphicspath{{./rsc/}{./rsc/pdf/}{./rsc/svg/}}

% Define mathbfit
\DeclareMathAlphabet{\mathbfit}{OML}{cmm}{b}{it}
\DeclareMathAlphabet{\mathbfsf}{\encodingdefault}{\sfdefault}{bx}{n}

% Define operators
\DeclareMathOperator*{\argmax}{arg\,max}
\DeclareMathOperator*{\argmin}{arg\,min}
\DeclareMathOperator*{\minimize}{Minimize}
\DeclareMathOperator*{\maximize}{Maximize}

\DeclareMathOperator*{\dotcup}{\dot{\cup}}

% Define textbfit
\makeatletter
\DeclareRobustCommand\bfseriesitshape{
  \not@math@alphabet\itshapebfseries\relax
  \fontseries\bfdefault
  \fontshape\itdefault
  \selectfont
}
\makeatother
\DeclareTextFontCommand{\textbfit}{\bfseriesitshape}

\expandafter\def\expandafter\normalsize\expandafter{%
    \normalsize
    \setlength\abovedisplayskip{15pt}
    \setlength\belowdisplayskip{15pt}
    \setlength\abovedisplayshortskip{15pt}
    \setlength\belowdisplayshortskip{15pt}
}

%
% The following commands set up the lecnum (lecture number)
% counter and make various numbering schemes work relative
% to the lecture number.
%
\newcounter{lecnum}
\renewcommand{\thepage}{\thelecnum-\arabic{page}}
\renewcommand{\thesection}{\thelecnum.\arabic{section}}
\renewcommand{\theequation}{\thelecnum.\arabic{equation}}
\renewcommand{\thefigure}{\thelecnum.\arabic{figure}}
\renewcommand{\thetable}{\thelecnum.\arabic{table}}

%
% Narrower margin
%
\def\changemargin#1#2{\list{}{\rightmargin#2\leftmargin#1}\item[]}
\let\endchangemargin=\endlist
%
% The following macro is used to generate the header.
%
\newcommand{\lecture}[4]{
  \pagestyle{myheadings}
  \thispagestyle{plain}
  \newpage
  \setcounter{lecnum}{#1}
  \setcounter{page}{1}
  \noindent
  \begin{center}
    \framebox{
      \vbox{
        \vspace{2mm}
        \hbox to 6.28in { {\bf MAT2013: Probability and Statistics~\cite{IPSUR-2010}~\cite{RP-Babatunde-2009} \hfill Spring 2017} }
        \vspace{4mm}
        \hbox to 6.28in { {\Large \hfill Lecture #1: #2  \hfill} }
        \vspace{2mm}
        \hbox to 6.28in { {\it Lecturer: #3 \hfill Scribes: #4\/} }
        \vspace{2mm}
      }
    }
  \end{center}
  \markboth{Lecture #1: #2}{Lecture #1: #2}
  {\textbf{Scribes}}: {
    Jisung Lim,
    \textit{B.S. Candidate of Industrial Engineering
    in Yonsei University, South Korea.}
  }

  {\textbf{Disclaimer}}: {
    \textit{These notes have not been subjected to the
    usual scrutiny reserved for formal publications.  They may be distributed
    outside this class only with the permission of the Instructor.}
  }
  \vspace*{4mm}
}
%
% Convention for citations is authors' initials followed by the year.
% For example, to cite a paper by Leighton and Maggs you would type
% \cite{LM89}, and to cite a paper by Strassen you would type \cite{S69}.
% (To avoid bibliography problems, for now we redefine the \cite command.)
% Also commands that create a suitable format for the reference list.
%\renewcommand{\cite}[1]{[#1]}
% \def\beginrefs{\begin{list}%
%         {[\arabic{equation}]}{\usecounter{equation}
%          \setlength{\leftmargin}{2.0truecm}\setlength{\labelsep}{0.4truecm}%
%          \setlength{\labelwidth}{1.6truecm}}}
% \def\endrefs{\end{list}}
% \def\bibentry#1{\item[\hbox{[#1]}]}

%Use this command for a figure; it puts a figure in wherever you want it.
%usage: \fig{NUMBER}{SPACE-IN-INCHES}{CAPTION}
\newcommand{\fig}[3]{
			\vspace{#2}
			\begin{center}
			Figure \thelecnum.#1:~#3
			\end{center}
	}
% Use these for theorems, lemmas, proofs, etc.

\theoremstyle{definition}
\newtheorem{definition}{Definition}[section]
\AtEndEnvironment{definition}{\qed}%
\newtheorem{theorem}{Theorem}[section]
\newtheorem{lemma}[theorem]{Lemma}
\newtheorem{proposition}[theorem]{Proposition}
\newtheorem{claim}[theorem]{Claim}
\newtheorem{corollary}[theorem]{Corollary}

\theoremstyle{remark}
\newtheorem{properties}[theorem]{Properties}
%\newtheorem{example}[theorem]{Example}
%\AtEndEnvironment{example}{\hfill\ensuremath{\Diamond}}%

\theoremstyle{remark}
\newtheorem{innerexample}[theorem]{Example}

\makeatletter
\patchcmd{\endinnerexample}{\endtrivlist}{\endlist}{}{}
\newenvironment{example}
 {\patchcmd{\@thm}{\trivlist}{\list{}{\leftmargin=3em \rightmargin=3em}}{}{}%
  \vspace*{10\p@}
  \innerexample\pushQED{\hfill\ensuremath{\Diamond}}}
 {\popQED\endinnerexample}
\makeatother


\newenvironment{prf}{{\bf Proof:}}{\hfill\rule{2mm}{2mm}}
\newenvironment{sol}{{\bf Solution:}}{\hfill\rule{2mm}{2mm}}
\newenvironment{skt}{{\bf Sketch:}}

% **** IF YOU WANT TO DEFINE ADDITIONAL MACROS FOR YOURSELF, PUT THEM HERE:

\newcommand\E{\mathbb{E}}

\begin{document}
%FILL IN THE RIGHT INFO.
%\lecture{**LECTURE-NUMBER**}{**DATE**}{**LECTURER**}{**SCRIBE**}
\lecture{3}{Distributions: Ideal Models}{Jae Guk, Kim}{Jisung Lim}
%\footnotetext{These notes are partially based on those of Nigel Mansell.}

% **** YOUR NOTES GO HERE:x
\justify
\section{Introduction}
Within the probabilistic framework, the ensemble, or aggregate behavior of the
random phenomenon in question is characterized by its \textbf{``probability distribution
function''} $f(x)$, or equivalently distribution function $F(x)$. In much the same
way that theoretical mathematical models are derived from `first-principles'
~\cite{ONLINE-1} for deterministic phenomena, it is also possible to derive these
theoretical PDFs as ideal models that describe our knoweledge of the underlying
random phenomena. \\[0.5\baselineskip]
Third of the lecture, `Distributions', is fully focused on developing and
analyzing ideal probability models, or equivalently distributions, of random
variability. We do this in each case by starting with all the relevant information
about the phenomenological mechanism behind the specific random variable, says $X$,
and we derive the expression for the PDF $f(x)$ appropriate to the random phenomenon
in question.

\section{Ideal Models of Discrete R.V.}

\subsection{The Discrete Uniform Random Variable}
\subsubsection{Basic Characteristics}
\subsubsection{Model Development}
\subsubsection{Mathematical Characteristics}

\subsection{The Bernoulli Random Variable}
\subsubsection{Basic Characteristics}
\subsubsection{Model Development}
\subsubsection{Mathematical Characteristics}

\subsection{The Binomial Random Variable}
\subsubsection{Basic Characteristics}
\subsubsection{Model Development}
\subsubsection{Mathematical Characteristics}

\subsection{The Negative Binomial Random Variable}
\subsubsection{Basic Characteristics}
\subsubsection{Model Development}
\subsubsection{Mathematical Characteristics}
a
\subsection{The Geometric Random Variable}
\subsubsection{Basic Characteristics}
\subsubsection{Model Development}
\subsubsection{Mathematical Characteristics}
a
\subsection{The Poisson Random Variable}
\subsubsection{Basic Characteristics}
\subsubsection{Model Development}
\subsubsection{Mathematical Characteristics}
a

\section{Ideal Models of Continuous R.V.}

\subsection{The Ratio Familly: The Uniform Random Variable}
\subsubsection{Basic Characteristics}
\subsubsection{Model Development}
\subsubsection{Mathematical Characteristics}

\subsection{The Gamma Familly: The Exponential Random Variable}
\subsubsection{Basic Characteristics}
\subsubsection{Model Development}
\subsubsection{Mathematical Characteristics}

\subsection{The Gaussian Familly: The Gaussian Random Variable}
\subsubsection{Basic Characteristics}
\subsubsection{Model Development}
\subsubsection{Mathematical Characteristics}


\printbibliography

% **** THIS ENDS THE EXAMPLES. DON'T DELETE THE FOLLOWING LINE:

\end{document}
