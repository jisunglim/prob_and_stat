\documentclass[twoside]{article}
\setlength{\oddsidemargin}{0 in}
\setlength{\evensidemargin}{0 in}
\setlength{\topmargin}{-0.6 in}
\setlength{\textwidth}{6.5 in}
\setlength{\textheight}{8.5 in}
\setlength{\headsep}{0.75 in}
\setlength{\parindent}{0 in}
\setlength{\parskip}{0.1 in}

%
% ADD PACKAGES here:
%
\usepackage[utf8]{inputenc}
\usepackage{amsmath,amsfonts,amssymb,graphicx}
\usepackage[document]{ragged2e}
\usepackage[Euler]{upgreek}
\usepackage{amsthm}

\usepackage[backend=biber]{biblatex}
\addbibresource{ref.bib}

\graphicspath{{./rsc/}{./rsc/pdf/}{./rsc/svg/}}

% Define mathbfit
\DeclareMathAlphabet{\mathbfit}{OML}{cmm}{b}{it}
\DeclareMathAlphabet{\mathbfsf}{\encodingdefault}{\sfdefault}{bx}{n}

% Define operators
\DeclareMathOperator*{\argmax}{arg\,max}
\DeclareMathOperator*{\argmin}{arg\,min}
\DeclareMathOperator*{\minimize}{Minimize}
\DeclareMathOperator*{\maximize}{Maximize}

% Define textbfit
\makeatletter
\DeclareRobustCommand\bfseriesitshape{
  \not@math@alphabet\itshapebfseries\relax
  \fontseries\bfdefault
  \fontshape\itdefault
  \selectfont
}
\makeatother
\DeclareTextFontCommand{\textbfit}{\bfseriesitshape}

%
% The following commands set up the lecnum (lecture number)
% counter and make various numbering schemes work relative
% to the lecture number.
%
\newcounter{lecnum}
\renewcommand{\thepage}{\thelecnum-\arabic{page}}
\renewcommand{\thesection}{\thelecnum.\arabic{section}}
\renewcommand{\theequation}{\thelecnum.\arabic{equation}}
\renewcommand{\thefigure}{\thelecnum.\arabic{figure}}
\renewcommand{\thetable}{\thelecnum.\arabic{table}}

%
% Narrower margin
%
\def\changemargin#1#2{\list{}{\rightmargin#2\leftmargin#1}\item[]}
\let\endchangemargin=\endlist
%
% The following macro is used to generate the header.
%
\newcommand{\lecture}[4]{
  \pagestyle{myheadings}
  \thispagestyle{plain}
  \newpage
  \setcounter{lecnum}{#1}
  \setcounter{page}{1}
  \noindent
  \begin{center}
    \framebox{
      \vbox{
        \vspace{2mm}
        \hbox to 6.28in { {\bf MAT2013: Probability and Statistics \hfill Spring 2017} }
        \vspace{4mm}
        \hbox to 6.28in { {\Large \hfill Problem Set #1  \hfill} }
        \vspace{2mm}
        \hbox to 6.28in { {\hfill \hfill Name: #2\/} }
        \vspace{2mm}
      }
    }
  \end{center}
  \vspace*{4mm}
}
%
% Convention for citations is authors' initials followed by the year.
% For example, to cite a paper by Leighton and Maggs you would type
% \cite{LM89}, and to cite a paper by Strassen you would type \cite{S69}.
% (To avoid bibliography problems, for now we redefine the \cite command.)
% Also commands that create a suitable format for the reference list.
%\renewcommand{\cite}[1]{[#1]}
% \def\beginrefs{\begin{list}%
%         {[\arabic{equation}]}{\usecounter{equation}
%          \setlength{\leftmargin}{2.0truecm}\setlength{\labelsep}{0.4truecm}%
%          \setlength{\labelwidth}{1.6truecm}}}
% \def\endrefs{\end{list}}
% \def\bibentry#1{\item[\hbox{[#1]}]}

%Use this command for a figure; it puts a figure in wherever you want it.
%usage: \fig{NUMBER}{SPACE-IN-INCHES}{CAPTION}
\newcommand{\fig}[3]{
			\vspace{#2}
			\begin{center}
			Figure \thelecnum.#1:~#3
			\end{center}
	}
% Use these for theorems, lemmas, proofs, etc.

\theoremstyle{definition}
\newtheorem{definition}{Definition}[section]
\newtheorem{theorem}{Theorem}[section]
\newtheorem{lemma}[theorem]{Lemma}
\newtheorem{proposition}[theorem]{Proposition}
\newtheorem{claim}[theorem]{Claim}
\newtheorem{corollary}[theorem]{Corollary}

\theoremstyle{remark}
\newtheorem{properties}[theorem]{Properties}
\newtheorem{example}[theorem]{Example}

\newenvironment{prf}{{\bf Proof:}}{\hfill\rule{2mm}{2mm}}
\newenvironment{sol}{{\bf Solution:}}{\hfill\rule{2mm}{2mm}}
\newenvironment{skt}{{\bf Sketch:}}

% **** IF YOU WANT TO DEFINE ADDITIONAL MACROS FOR YOURSELF, PUT THEM HERE:

\newcommand\E{\mathbb{E}}

\begin{document}
%FILL IN THE RIGHT INFO.
%\lecture{**LECTURE-NUMBER**}{**DATE**}{**LECTURER**}{**SCRIBE**}
\lecture{3}{Jisung Lim (2014147040)}
%\footnotetext{These notes are partially based on those of Nigel Mansell.}

% **** YOUR NOTES GO HERE:x
\section{Solution}
\justify
\begin{enumerate}
  \item Consider a random variable vector $(X, Y)$ with joint pdf
  \begin{equation*}
    f(x, y) = \left\{
    \begin{array}{ll}
      e^{-y}, \quad & 0 < x < y < \infty \\
      0,      \quad & \textrm{otherwise}
    \end{array}
    \right.
  \end{equation*}
  \begin{enumerate}
    \item Compute $P(X + Y \geq 1)$.\\
    \begin{sol}\\
      \begin{equation*}
        \begin{split}
          P(X+Y \geq 1)
          &= 1 - P(X+Y < 1) \\
          &= 1 - \iint_{\mathbb{R}^2} f(x,y) \; dx dy \\
          &= 1 - \int_{0}^{\frac{1}{2}} \int_{x}^{1-x} e^{-y} \; dy dx \\
          &= 1 - \int_{0}^{\frac{1}{2}} \left. -e^{-y} \right|_{x}^{1-x}\; dx \\
          &= 1 - \int_{0}^{\frac{1}{2}} -e^{x-1} + e^{-x}\; dx \\
          &= 1 - \left. (-e^{x-1} - e^{-x}) \right|_{0}^{\frac{1}{2}} \\
          &= 1 - \left( -2e^{-1/2} + e^{-1} + 1 \right) \\
          &= 2e^{-1/2} - e^{-1}
        \end{split}
      \end{equation*}
    \end{sol}

    \item Find the marginal pdfs $f_X$ and $f_Y$. \\
    \begin{sol} \\
      If $x < 0$, then $f_X(x) = 0$. If $x \geq 0$, then
      \begin{equation*}
        \begin{split}
          f_X(x)
          &= \int_{x}^{\infty} e^{-y}\; dy \\
          &= \lim_{t\rightarrow 0} \left. -e^{-y}\right|_{x}^{t} \\
          &= \lim_{t\rightarrow 0} e^{-x} - e^{-t} \\
          &= e^{-x}
        \end{split}
      \end{equation*}

      If $y < 0$, then $f_Y(y) = 0$. If $y \geq 0$, then
      \begin{equation*}
        \begin{split}
          f_Y(y)
          &= \int_{0}^{y} e^{-y}\; dx \\
          &= y e^{-y}
        \end{split}
      \end{equation*}
    \end{sol}
  \end{enumerate}

  \clearpage
  \item Let $X_i$, $i = 1, 2, \ldots$ be independent exponential random variables
  with rate $\eta_i$. Let $Z = \min \{X_1, X_2, \ldots,$ $X_n\}$ and $Y = \max \{
  X_1, X_2, \ldots, X_n \}$. Find the distributions of $Z$ and $Y$. \\
  \begin{sol} \\
    For the random variable $X_i$, the probability distribution function $f^{(i)}(x)$ is given
    by
    \begin{equation*}
      f^{(i)}(x) = \eta_i e^{-\eta_i x} \quad \forall x \in (0, \infty),
    \end{equation*}
    and the distribution function $F^{(i)}(x)$ is given by
    \begin{equation*}
      F^{(i)}(x)
      = \int_{0}^{x} \eta_i e^{-\eta_i t} \;dt
      = \left. -e^{-\eta_i t} \right|_{0}^{x}
      = 1 - e^{-\eta_i x}
    \end{equation*}
    Now, let the random variable $Z$ is given by
    \begin{equation*}
      Z = \min{ \{ X_1, X_2, \ldots, X_n \} },
    \end{equation*}
    then distribution function of $Z$, says $F_Z(z)$ is given by
    \begin{equation*}
      \begin{split}
        F_Z(z)
        &= P(Z \leq z) = \mathbb{P}[\{ \omega \in \Omega : Z(\omega) \leq z \}] \\
        &= \mathbb{P}[\{ Z \leq z \}] \\
        &= \mathbb{P}[\{ \min{ \{ X_1, X_2, \ldots, X_n \} } \leq z \}] \\
        &= \mathbb{P}[\{X_1 \leq z\} \cup \{X_2 \leq z\} \cup \cdots \cup \{X_n \leq z\}] \\
        &= \mathbb{P}[\Omega \setminus (\{X_1 > z\} \cap \{X_2 > z\} \cap \cdots \cap \{X_n > z\})] \\
        &= 1 - \mathbb{P}[\{X_1 > z\} \cap \{X_2 > z\} \cap \cdots \cap \{X_n > z\}] \\
      \end{split}
    \end{equation*}
    Since the random variable $X_1, \ldots, X_n$ are mutually independent, then
    for $A_i$ in $\sigma$-algebra $\mathcal{A}$, every events ${\{\omega \in \Omega : X_i(\omega) \in A_i\}}$
    are mutually independent. Therefore,
    \begin{equation*}
      \begin{split}
        F_Z(z)
        &= 1 - \mathbb{P}[\{X_1 > z\} \cap \{X_2 > z\} \cap \cdots \cap \{X_n > z\}] \\
        &= 1 - \mathbb{P}[\{X_1 > z\}] \mathbb{P}[\{X_2 > z\}] \cdots \mathbb{P}[\{X_n > z\}] \quad \textrm{(by independence)}\\
        &= 1 - P_{X_1}(X_1 > z) P_{X_2}(X_2 > z) \cdots P_{X_n}(X_n > z) \\
      \end{split}
    \end{equation*}
    Since $P_{X_i}(X_i > z) = 1 - P_{X_i}(X_i \leq z) = 1 - F^{(i)}(z) = e^{-\eta_i z}$, then
    \begin{equation*}
      \begin{split}
        F_Z(z)
        &= 1 - P_{X_1}(X_1 > z) P_{X_2}(X_2 > z) \cdots P_{X_n}(X_n > z) \\
        &= 1 - e^{-\eta_1 z}e^{-\eta_2 z} \cdots e^{-\eta_n z} \\
        &= 1 - e^{-(\eta_1 + \eta_2 + \cdots + \eta_n) z}
      \end{split}
    \end{equation*}
    Hence,
    \begin{equation*}
      F_Z(z) = 1 - e^{-(\eta_1 + \eta_2 + \cdots + \eta_n) z}
    \end{equation*}

    For $Y$, in the same manner,
    \begin{equation*}
      \begin{split}
        F_Y(y)
        &= P_Y(Y \leq y) = \mathbb{P}[\{ Y \leq y \}] \\
        &= \mathbb{P}[\{ \max{ \{ X_1, X_2, \ldots, X_n \} } \leq  y \}] \\
        &= \mathbb{P}[\{X_1 \leq y\} \cap \{X_2 \leq y\} \cap \cdots \cap \{X_n \leq z\}] \\
        &= \mathbb{P}[\{X_1 \leq y\}] \mathbb{P}[\{X_2 \leq y\}] \cdots \mathbb{P}[\{X_n \leq z\}] \\
        &= P_{X_1}(X_1 \leq y) P_{X_2}(X_2 \leq y) \cdots P_{X_n}(X_n \leq y) \\
        &= (1 - e^{-\eta_1 y})(1 - e^{-\eta_2 y})\cdots(1 - e^{-\eta_n y}) \\
      \end{split}
    \end{equation*}
    Hence,
    \begin{equation*}
      F_Y(y) = (1 - e^{-\eta_1 y})(1 - e^{-\eta_2 y})\cdots(1 - e^{-\eta_n y})
    \end{equation*}
  \end{sol}

  \clearpage
  \item Prove the following statements:
  \begin{enumerate}
    \item $Cov(X, Y) = Cov(Y, X)$ \\
    \begin{sol}\\
      \begin{equation*}
        \begin{split}
          Cov(X, Y)
          &= \mathbb{E}[(X-\mu_X)(Y - \mu_Y)] \\
          &= \mathbb{E}[(Y-\mu_Y)(X - \mu_X)] \\
          &= Cov(Y, X)
        \end{split}
      \end{equation*}
    \end{sol}
    \item $Cov(X, X) = Var(X)$ \\
    \begin{sol}\\
      \begin{equation*}
        \begin{split}
          Cov(X, Y)
          &= \mathbb{E}[(X-\mu_X)(X - \mu_X)] \\
          &= \mathbb{E}[{(X - \mu_X)}^2] \\
          &= Var(X)
        \end{split}
      \end{equation*}
    \end{sol}
    \item $Cov(aX, Y) = aCov(X, Y)$ \\
    \begin{sol}\\
      \begin{equation*}
        \begin{split}
          Cov(aX, Y)
          &= \mathbb{E}[(aX-\mathbb{E}[aX])(Y - \mu_Y)] \\
          &= \mathbb{E}[(aX-a\mathbb{E}[X])(Y - \mu_Y)] \\
          &= a\mathbb{E}[(X-\mu_X)(Y - \mu_Y)] \\
          &= aCov(X, Y)
        \end{split}
      \end{equation*}
    \end{sol}
    \item $Cov(\sum_{i=1}^{n} X_i, \sum_{i=1}^{n} Y_i) = \sum_{i=1}^{n} \sum_{i=1}^{n} Cov(X_i, Y_i)$ \\
    \begin{sol}\\
      \begin{equation*}
        \begin{split}
          Cov\left(\sum_{i=1}^{n} X_i, \sum_{i=1}^{n} Y_i\right)
          &= \mathbb{E}[(\sum_{i=1}^{n} X_i-\mathbb{E}[\sum_{i=1}^{n} X_i])(\sum_{i=1}^{n} Y_i - \mathbb{E}[\sum_{i=1}^{n} Y_i])] \\
          &= \mathbb{E}[(\sum_{i=1}^{n} X_i-\sum_{i=1}^{n} \mathbb{E}[X_i])(\sum_{i=1}^{n} Y_i - \sum_{i=1}^{n} \mathbb{E}[Y_i])] \\
          &= \mathbb{E}[\sum_{i=1}^{n} (X_i - \mathbb{E}[X_i])\sum_{i=1}^{n} (Y_i - \mathbb{E}[Y_i])] \\
          &= \mathbb{E}[\sum_{i=1}^{n} \sum_{i=1}^{n} (X_i - \mathbb{E}[X_i]) (Y_i - \mathbb{E}[Y_i])] \\
          &= \sum_{i=1}^{n} \sum_{i=1}^{n} \mathbb{E}[(X_i - \mathbb{E}[X_i]) (Y_i - \mathbb{E}[Y_i])] \\
          &= \sum_{i=1}^{n} \sum_{i=1}^{n} Cov(X_i, Y_i)
        \end{split}
      \end{equation*}
    \end{sol}
  \end{enumerate}
  \clearpage
  \item Suppose that $X$ and $Y$ are independent continuous random variables. Find
  the distribution of $X+Y$. \\
  \begin{sol}\\
    Let $Z$ be a random variable given by $Z = X + Y$. Then a distribution function
    of $Z$, says $F_Z(z)$, is given by
    \begin{equation*}
      \begin{split}
        F_{Z}(z)
        &= P_Z(Z \geq z) = \mathbb{P}[\{ Z \geq z \}] \\
        &= \mathbb{P}[\{ \omega \in \Omega : X(\omega) + Y(\omega) \geq z \}] \\
        &= \int_{-\infty}^{\infty} \mathbb{P}[\{ X \geq z - k \} \cap \{Y \geq k\}] \;dk \\
      \end{split}
    \end{equation*}
    Since the random variable $X$ and $Y$ are mutually independent, then two events
    ${\{\omega \in \Omega : X(\omega) \in A\}}$ and ${\{\omega \in \Omega : Y(\omega)
    \in B\}}$ are mutually independent where $A$ and $B$ in $\sigma$-algebra
    $\mathcal{A}$. Therefore,
    \begin{equation*}
      \begin{split}
        F_{Z}(z)
        &= \int_{-\infty}^{\infty} \mathbb{P}[\{ X \geq z - k \} \cap \{Y \geq k\}] \;dk \\
        &= \int_{-\infty}^{\infty} \mathbb{P}[\{ X \geq z - k \}]\mathbb{P}[\{Y \geq k\}] \;dk \\
        &= \int_{-\infty}^{\infty} P_X(X \geq z - k)P_Y(Y \geq k) \;dk \\
        &= \int_{-\infty}^{\infty} F_X(z - k)F_Y(k) \;dk \\
      \end{split}
    \end{equation*}
  \end{sol}
\end{enumerate}



\printbibliography

% **** THIS ENDS THE EXAMPLES. DON'T DELETE THE FOLLOWING LINE:

\end{document}
