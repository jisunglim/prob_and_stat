\documentclass[twoside]{article}
\setlength{\oddsidemargin}{0 in}
\setlength{\evensidemargin}{0 in}
\setlength{\topmargin}{-0.6 in}
\setlength{\textwidth}{6.5 in}
\setlength{\textheight}{8.5 in}
\setlength{\headsep}{0.75 in}
\setlength{\parindent}{0 in}
\setlength{\parskip}{0.1 in}

%
% ADD PACKAGES here:
%
\usepackage[utf8]{inputenc}
\usepackage{amsmath,amsfonts,amssymb,graphicx}
\usepackage[document]{ragged2e}
\usepackage[Euler]{upgreek}
\usepackage{amsthm}

\usepackage[backend=biber]{biblatex}
\addbibresource{ref.bib}

\graphicspath{{./rsc/}{./rsc/pdf/}{./rsc/svg/}}

% Define mathbfit
\DeclareMathAlphabet{\mathbfit}{OML}{cmm}{b}{it}
\DeclareMathAlphabet{\mathbfsf}{\encodingdefault}{\sfdefault}{bx}{n}

% Define operators
\DeclareMathOperator*{\argmax}{arg\,max}
\DeclareMathOperator*{\argmin}{arg\,min}
\DeclareMathOperator*{\minimize}{Minimize}
\DeclareMathOperator*{\maximize}{Maximize}

% Define textbfit
\makeatletter
\DeclareRobustCommand\bfseriesitshape{
  \not@math@alphabet\itshapebfseries\relax
  \fontseries\bfdefault
  \fontshape\itdefault
  \selectfont
}
\makeatother
\DeclareTextFontCommand{\textbfit}{\bfseriesitshape}

%
% The following commands set up the lecnum (lecture number)
% counter and make various numbering schemes work relative
% to the lecture number.
%
\newcounter{lecnum}
\renewcommand{\thepage}{\thelecnum-\arabic{page}}
\renewcommand{\thesection}{\thelecnum.\arabic{section}}
\renewcommand{\theequation}{\thelecnum.\arabic{equation}}
\renewcommand{\thefigure}{\thelecnum.\arabic{figure}}
\renewcommand{\thetable}{\thelecnum.\arabic{table}}

%
% Narrower margin
%
\def\changemargin#1#2{\list{}{\rightmargin#2\leftmargin#1}\item[]}
\let\endchangemargin=\endlist
%
% The following macro is used to generate the header.
%
\newcommand{\lecture}[4]{
  \pagestyle{myheadings}
  \thispagestyle{plain}
  \newpage
  \setcounter{lecnum}{#1}
  \setcounter{page}{1}
  \noindent
  \begin{center}
    \framebox{
      \vbox{
        \vspace{2mm}
        \hbox to 6.28in { {\bf MAT2013: Probability and Statistics \hfill Spring 2017} }
        \vspace{4mm}
        \hbox to 6.28in { {\Large \hfill Problem Set #1  \hfill} }
        \vspace{2mm}
        \hbox to 6.28in { {\hfill \hfill Name: #2\/} }
        \vspace{2mm}
      }
    }
  \end{center}
  \vspace*{4mm}
}
%
% Convention for citations is authors' initials followed by the year.
% For example, to cite a paper by Leighton and Maggs you would type
% \cite{LM89}, and to cite a paper by Strassen you would type \cite{S69}.
% (To avoid bibliography problems, for now we redefine the \cite command.)
% Also commands that create a suitable format for the reference list.
%\renewcommand{\cite}[1]{[#1]}
% \def\beginrefs{\begin{list}%
%         {[\arabic{equation}]}{\usecounter{equation}
%          \setlength{\leftmargin}{2.0truecm}\setlength{\labelsep}{0.4truecm}%
%          \setlength{\labelwidth}{1.6truecm}}}
% \def\endrefs{\end{list}}
% \def\bibentry#1{\item[\hbox{[#1]}]}

%Use this command for a figure; it puts a figure in wherever you want it.
%usage: \fig{NUMBER}{SPACE-IN-INCHES}{CAPTION}
\newcommand{\fig}[3]{
			\vspace{#2}
			\begin{center}
			Figure \thelecnum.#1:~#3
			\end{center}
	}
% Use these for theorems, lemmas, proofs, etc.

\theoremstyle{definition}
\newtheorem{definition}{Definition}[section]
\newtheorem{theorem}{Theorem}[section]
\newtheorem{lemma}[theorem]{Lemma}
\newtheorem{proposition}[theorem]{Proposition}
\newtheorem{claim}[theorem]{Claim}
\newtheorem{corollary}[theorem]{Corollary}

\theoremstyle{remark}
\newtheorem{properties}[theorem]{Properties}
\newtheorem{example}[theorem]{Example}

\newenvironment{prf}{{\bf Proof:}}{\hfill\rule{2mm}{2mm}}
\newenvironment{sol}{{\bf Solution:}}{\hfill\rule{2mm}{2mm}}
\newenvironment{skt}{{\bf Sketch:}}

% **** IF YOU WANT TO DEFINE ADDITIONAL MACROS FOR YOURSELF, PUT THEM HERE:

\newcommand\E{\mathbb{E}}

\begin{document}
%FILL IN THE RIGHT INFO.
%\lecture{**LECTURE-NUMBER**}{**DATE**}{**LECTURER**}{**SCRIBE**}
\lecture{1}{Jisung Lim (2014147040)}
%\footnotetext{These notes are partially based on those of Nigel Mansell.}

% **** YOUR NOTES GO HERE:x
\section{1. Prove that if $P(\cdot)$ is a legitimate probability function and $B$
is a set with $P(B) > 0$, then $P(\cdot|B)$ also satisfies Kolmogorov's Axioms.}
\begin{skt}
  $$
  P(A|B) \underbrace{=}_{Def.} \frac{P(A \cap B)}{P(B)}
  \underbrace{\geq}_{P(A) \geq P(A \cap B)} \frac{P(A)}{P(B)}
  \underbrace{\geq}_{P(A) \geq 0,\,P(B) > 0} 0
  $$
  $$
  P(\Omega|B) \underbrace{=}_{Def.} \frac{P(\Omega \cap B)}{P(B)}
  \underbrace{=}_{P(\Omega \cap B) = P(B)} \frac{P(B)}{P(B)} = 1
  $$
  $$
  P(\bigcup_{i=1}^{\infty} A_i | B)
  \underbrace{=}_{Def.} \frac{P((\bigcup_{i=1}^{\infty} A_i) \cap B)}{P(B)}
  \underbrace{=}_{Dist. law} \frac{P(\bigcup_{i=1}^{\infty} (A_i \cap B))}{P(B)}
  \underbrace{=}_{Axiom (3)} \sum_{i=1}^{\infty} \frac{P(A_i \cap B)}{P(B)}
  \underbrace{=}_{Def.} \sum_{i=1}^{\infty} P(A_i|B)
  $$
\end{skt}
\begin{prf} \\
  Let $(\Omega, \mathcal{A}, P)$ be a probability space, then the probability
  measure $P$ satisfies following axioms:
  \begin{enumerate}
    \item $P: \mathcal{A} \rightarrow [0,\infty)$ \quad
          (i.e. $\forall A \in \mathcal{A}, \quad P(A) \geq 0$)
    \item $P(\Omega) = 1$
    \item If $A_i$ is a sequence of disjoint sets that belongs to $\mathcal{A}$,
          then
          $$
          P(\bigcup_{i=1}^{\infty} A_i) = \sum_{i=1}^{\infty} P(A_i)
          $$
  \end{enumerate}

\end{prf}


\printbibliography

% **** THIS ENDS THE EXAMPLES. DON'T DELETE THE FOLLOWING LINE:

\end{document}
