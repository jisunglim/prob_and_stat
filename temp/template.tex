\documentclass[twoside]{article}
\setlength{\oddsidemargin}{0 in}
\setlength{\evensidemargin}{0 in}
\setlength{\topmargin}{-0.6 in}
\setlength{\textwidth}{6.5 in}
\setlength{\textheight}{8.5 in}
\setlength{\headsep}{0.75 in}
\setlength{\parindent}{0 in}
\setlength{\parskip}{0.1 in}

%
% ADD PACKAGES here:
%
\usepackage[utf8]{inputenc}
\usepackage{amsmath,amsfonts,amssymb,graphicx}
\usepackage[document]{ragged2e}
\usepackage[Euler]{upgreek}
\usepackage{amsthm}

\usepackage[backend=biber]{biblatex}
\addbibresource{ref.bib}

\graphicspath{{./rsc/}{./rsc/pdf/}{./rsc/svg/}}

% Define mathbfit
\DeclareMathAlphabet{\mathbfit}{OML}{cmm}{b}{it}
\DeclareMathAlphabet{\mathbfsf}{\encodingdefault}{\sfdefault}{bx}{n}

% Define operators
\DeclareMathOperator*{\argmax}{arg\,max}
\DeclareMathOperator*{\argmin}{arg\,min}
\DeclareMathOperator*{\minimize}{Minimize}
\DeclareMathOperator*{\maximize}{Maximize}

% Define textbfit
\makeatletter
\DeclareRobustCommand\bfseriesitshape{
  \not@math@alphabet\itshapebfseries\relax
  \fontseries\bfdefault
  \fontshape\itdefault
  \selectfont
}
\makeatother
\DeclareTextFontCommand{\textbfit}{\bfseriesitshape}

%
% The following commands set up the lecnum (lecture number)
% counter and make various numbering schemes work relative
% to the lecture number.
%
\newcounter{lecnum}
\renewcommand{\thepage}{\thelecnum-\arabic{page}}
\renewcommand{\thesection}{\thelecnum.\arabic{section}}
\renewcommand{\theequation}{\thelecnum.\arabic{equation}}
\renewcommand{\thefigure}{\thelecnum.\arabic{figure}}
\renewcommand{\thetable}{\thelecnum.\arabic{table}}

%
% Narrower margin
%
\def\changemargin#1#2{\list{}{\rightmargin#2\leftmargin#1}\item[]}
\let\endchangemargin=\endlist
%
% The following macro is used to generate the header.
%
\newcommand{\lecture}[4]{
  \pagestyle{myheadings}
  \thispagestyle{plain}
  \newpage
  \setcounter{lecnum}{#1}
  \setcounter{page}{1}
  \noindent
  \begin{center}
    \framebox{
      \vbox{
        \vspace{2mm}
        \hbox to 6.28in { {\bf MAT2013: Probability and Statistics \hfill Spring 2017} }
        \vspace{4mm}
        \hbox to 6.28in { {\Large \hfill Problem Set #1  \hfill} }
        \vspace{2mm}
        \hbox to 6.28in { {\hfill \hfill Name: #2\/} }
        \vspace{2mm}
      }
    }
  \end{center}
  \vspace*{4mm}
}
%
% Convention for citations is authors' initials followed by the year.
% For example, to cite a paper by Leighton and Maggs you would type
% \cite{LM89}, and to cite a paper by Strassen you would type \cite{S69}.
% (To avoid bibliography problems, for now we redefine the \cite command.)
% Also commands that create a suitable format for the reference list.
%\renewcommand{\cite}[1]{[#1]}
% \def\beginrefs{\begin{list}%
%         {[\arabic{equation}]}{\usecounter{equation}
%          \setlength{\leftmargin}{2.0truecm}\setlength{\labelsep}{0.4truecm}%
%          \setlength{\labelwidth}{1.6truecm}}}
% \def\endrefs{\end{list}}
% \def\bibentry#1{\item[\hbox{[#1]}]}

%Use this command for a figure; it puts a figure in wherever you want it.
%usage: \fig{NUMBER}{SPACE-IN-INCHES}{CAPTION}
\newcommand{\fig}[3]{
			\vspace{#2}
			\begin{center}
			Figure \thelecnum.#1:~#3
			\end{center}
	}
% Use these for theorems, lemmas, proofs, etc.

\theoremstyle{definition}
\newtheorem{definition}{Definition}[section]
\newtheorem{theorem}{Theorem}[section]
\newtheorem{lemma}[theorem]{Lemma}
\newtheorem{proposition}[theorem]{Proposition}
\newtheorem{claim}[theorem]{Claim}
\newtheorem{corollary}[theorem]{Corollary}

\theoremstyle{remark}
\newtheorem{properties}[theorem]{Properties}
\newtheorem{example}[theorem]{Example}

\newenvironment{prf}{{\bf Proof:}}{\hfill\rule{2mm}{2mm}}
\newenvironment{sol}{{\bf Solution:}}{\hfill\rule{2mm}{2mm}}
\newenvironment{skt}{{\bf Sketch:}}

% **** IF YOU WANT TO DEFINE ADDITIONAL MACROS FOR YOURSELF, PUT THEM HERE:

\newcommand\E{\mathbb{E}}

\begin{document}
%FILL IN THE RIGHT INFO.
%\lecture{**LECTURE-NUMBER**}{**DATE**}{**LECTURER**}{**SCRIBE**}
\lecture{3}{Jisung Lim (2014147040)}
%\footnotetext{These notes are partially based on those of Nigel Mansell.}

% **** YOUR NOTES GO HERE:x
\section{Solution}
\justify

\textbf{1. Exisiting setting} \\[0.5\baselineskip]
There exists $n$ users which is denoted by
\begin{equation}
  u_1, u_2, \ldots, u_n,
\end{equation}
and a list of enemies of $u_i$, $\mathcal{E}(u_i)$, briefly $\mathcal{E}_i$, is given.
Let $\mathcal{F}(u_i)$, briefly $\mathcal{F}_i$,  is a list of friends of $u_i$, then we have three conditions as
follows

\begin{itemize}
  \item Condition 1. \\
  \begin{equation}
    \forall i, j (\neq i) \in \{1, \ldots, n\}, \quad u_i \in \{\mathcal{E}_j\} \quad \textit{iff} \quad u_j \in \{\mathcal{E}_i\}
  \end{equation}
  \item Condition 2. \\
  \begin{equation}
    \forall i \in \{1, \ldots, n\}, \quad u_i \notin \{\mathcal{E}_i\}
  \end{equation}
  \item Condition 3. \\
  For all $i, j (\neq i) \in \{1, \ldots, n\}$, $u_i$ and $u_j$ are friend \textit{iff}
  \begin{equation}
    \exists u_k \in \{\mathcal{E}_i\} \cap \{\mathcal{E}_j\}
  \end{equation}
\end{itemize}

\textbf{2. Define a matrix} \\[0.5\baselineskip]
For all $i,j \in \{1, \ldots, n\}$ let $E$ and $F$ be $n \times n$ matrices
which is given by
\begin{equation}
  {(E)}_{ij} = \left\{
  \begin{array}{ll}
    1, \quad \textrm{if $u_i$ and $u_j$ are enemy}\\
    0, \quad \textrm{otherwise}
  \end{array}
  \right.,
\end{equation}
and
\begin{equation}
  {(F)}_{ij} = \left\{
  \begin{array}{ll}
    1, \quad \textrm{if $u_i$ and $u_j$ are friend}\\
    0, \quad \textrm{otherwise}
  \end{array}
  \right..
\end{equation}

By condition 1 and 2, the matrix $E$ is a symmetry matrix (i.e., $(E)_{ij} = (E)_{ji} \;\forall i, j$)
and has zero diagonal entries (i.e. $E_{ii} = 0 \;\forall i$),
hence column vectors $\mathbf{c}_j(E)$ and row vectors $\mathbf{r}_i(E)$,
briefly $\mathbf{c}_j$ and $\mathbf{r}_i$, satisfies
\begin{equation}
  i = j \Rightarrow \mathbf{c}_j = \mathbf{r}_i.
\end{equation}

Now, let us consider the inner product of $\mathbf{r}_i$ and $\mathbf{c}_j$, which
is given by
\begin{equation}
  \mathbf{r}_i \cdot \mathbf{c}_j
  = \sum_{k = i}^{n} (\mathbf{r}_i)_k (\mathbf{c}_j)_k
  = \sum_{k = i}^{n} (E)_{ik} (E)_{kj}
  = \sum_{k = i}^{n} \mathbb{I}_k(i, j)
\end{equation}
where indicator function $\mathbb{I}_k(i, j)$ is
\begin{equation}
  \mathbb{I}_k(i, j) = \left\{
  \begin{array}{ll}
    1, \quad \exists u_k \in \{\mathcal{E}_i\} \cap \{\mathcal{E}_j\}\\
    0, \quad \textrm{otherwise}
  \end{array}
  \right.,
\end{equation}
then we can say that
\begin{equation}
  \mathbf{r}_i \cdot \mathbf{c}_j = \left\{
  \begin{array}{ll}
    l_{ij}, \quad & \{\mathcal{E}_i\} \cap \{\mathcal{E}_j\} \neq \emptyset \\
    0, \quad & \{\mathcal{E}_i\} \cap \{\mathcal{E}_j\} = \emptyset
  \end{array}
  \right.
\end{equation}
where $0 < l_{ij} < n$.

\textbf{3. Find the matrix $F$ from $E$} \\[0.5\baselineskip]
Let a $n \times n$ matrix $M = EE$, then, by (3.10),
\begin{equation}
  {(M)}_{ij} = \mathbf{r}_i \cdot \mathbf{c}_j = \left\{
  \begin{array}{ll}
    l_{ij}, \quad & \{\mathcal{E}_i\} \cap \{\mathcal{E}_j\} \neq \emptyset \\
    0, \quad & \{\mathcal{E}_i\} \cap \{\mathcal{E}_j\} = \emptyset
  \end{array}
  \right.
\end{equation}
where $0 < l_{ij} < n$. By condition 3,
\begin{equation}
  {(M)}_{ij} = \left\{
  \begin{array}{ll}
    l_ij, \quad \textrm{if $u_i$ and $u_j$ are friend}\\
    0,    \quad \textrm{if $u_i$ and $u_j$ are not friend}
  \end{array}
  \right., \quad \forall i, j(\neq i).
\end{equation}

Be careful that if $i = j$, there may be inconsistent result such that
\begin{equation}
  {(M)}_{ii} = \left\{
  \begin{array}{ll}
    0, \quad & \textrm{if}\; {\mathcal{E}_i} = \emptyset \\
    l_{ii}, \quad & \textrm{otherwise}
  \end{array}
  \right.
\end{equation}

Hence, if we assume that every user $u_i$ is friend of itself, then we can
find the matrix $F$ as follows
\begin{equation}
  {(F)}_{ij} = \left\{
  \begin{array}{ll}
    1, \quad & i = j \\
    0, \quad & {(M)}_{ij} = 0 \\
    {(M)}_{ij}/l_{ij}, \quad & {(M)}_{ij} = l_{ij}
  \end{array}
  \right.
\end{equation}

\printbibliography

% **** THIS ENDS THE EXAMPLES. DON'T DELETE THE FOLLOWING LINE:

\end{document}
